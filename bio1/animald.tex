\chapter{Animal defense mechanisms (I)}

\section{General defense mechanisms}
%% general information for these can be found in webpages just make sure they look credible

\paragraph{Chemical defenses}
\footnote{from https://www.sciencedirect.com/topics/agricultural-and-biological-sciences/chemical-defense}
Chemical defenses are widespread within different groups.
Comparing two major animal groups, chemical defenses are more extensive in invertebrates.
These chemical defenses include toxins, or chemical with noxious odors, which affect the sensory systems some predators use to capture prey.
Chemical defenses, similar to plant secondary metabolites, are synthesized in the bodies of animals.
An example of a chemical defense is the secretion of ink of a cephalopod when it is threatened.
\footnote{https://www.theguardian.com/science/2017/aug/09/why-do-cephalopods-produce-ink-and-what-on-earth-is-it-anyway}
This results to a reduction of visibility (if the predator relies on light), protecting the animal.
There are accounts of usage of cephalopod ink for commercial sources.
\footnote{\textit{ibid.} as above}

\paragraph{Immune systems}
The human immune system is a very complex defense mechanism against unrecognized pathogens in the body.
Multiple organs work together in order to keep the unwanted pathogens away from the body.
Humans have two kinds of immunity: innate and adaptive immunity.
Innate immunity given by the presence of the skin, mucous membranes, antimicrobial substances, and natural killer cells \cite[p.396-398]{anaphy}.
Adaptive immunity on the other hand is an adaptation from the production of antigens that the system recognizes and terminates foreign material \cite[p.399-407]{anaphy}.

\paragraph{Predator evasion}
Predator-prey relationships are ubiquitous in nature.
Naturally, for the prey to survive, it must have mechanisms to stay away from predators in order to be considered fit.
An example of which is given above on ink secretion of some cephalopods.

\section{Theoretical study on defense phenotype variance}
In this section, a theoretical study by \citeA{Wang2019} on the evolution of defense phenotypes will be summarized and discussed.
The study revolves around the evolution of two defense phenotypes that organisms can use sequentially \textit{i.e.} one defense is executed after another given that a first one fails.
As summarized above animals have a wide variety of defense mechanisms to choose from and they are not limited to only one defense trait (as was seen with plants).
These traits may also have variety across the population, given the possibility of genetic drift \cite{Ridley}.
The study aimed to propose a potentially predictive explanation for kinds of defense variation.

\subsection{Model description}
Sequential traits ``must be crossed in turn'', therefore it is reasonable to assume that selection biasedly acts on the first defense trait executed by the organism.
In order to study the time evolution of traits, the following variables were introduced.

\begin{enumerate}
    \item defense phenotypes;
    \item average phenotypes;
    \item phenotype variances;
    \item total population
\end{enumerate}

Defense phenotypes and their variances were studied for each defense trait.
The total population was also held constant since the \emph{frequencies} of individuals with \emph{defense traits} are studied.
Ideal defense phenotypes were assumed, and that deviations around these ideals (tolerance) were accounted for.
Though it was not explicitly stated, it is likely assumed that only one agent is being acted upon by the defense traits.
The effectiveness of the phenotype is given as a constant value, which supports the assumption.
Variant fitness was based on the \emph{conditional effectiveness} of the defense phenotypes.
This relates indirectly to the evolutionary definition of fitness as an ability to leave healthy offspring \cite{Ridley}.
\footnote{It is reasonable to assume that if defense mechanisms did not work, then the variant cannot leave healthy progeny.}
And finally, mutation strengths were also considered.

The distribution of the effectivity (of phenotypes), mutations (of phenotypes), and phenotypes are assumed to be Gaussian (\emph{normally} distributed).

\subsection{Results}
Based on model simulations it was found that because of trait selection defense phenotypes moved towards the ideal phenotype.
These was caused by the following mechanisms:
1) mutation;
2) order of defense execution;
3) trait effectivity.

If there were no mutations in the population then all of the defense phenotypes were the same.
Stronger mutations also resulted to high phenotype variance.
As the traits evolve, it was found that the variance of the first trait is lower compared to the second, and this evolution is independent of conditional fitness.
\footnote{Though this likely to be a circular argument since the equation was modeled such that this selection is biased towards the mutation of the first defense trait executed.}
It was also observed that as a the effectivity of a defense trait increases, its variation decreases, while the variation of the other trait increases, regardless of execution order.
\footnote{With the previous sentence, an interplay of effectivity, mutation, and variance results from the complex system.}
