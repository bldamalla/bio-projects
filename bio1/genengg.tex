\chapter{Genetic engineering (I)}
%% all citations are from \citeA{genengg}

The genetic disease to be discussed in this chapter is the \textalpha -1-antitrypsin (AAT) deficiency.
As the name implies, it is caused by a deficiency of properly functioning AAT.
The causes of the deficiency and its treatment, augmentation therapy and gene therapy are discussed.
Most of the information in this chapter is retrieved from the review of \citeA{genengg} on the assessment and treatment of the disease.

\section{\textalpha -1-antitrypsin and the deficiency}
\subsection{\textalpha -1-antitrypsin (AAT)}
AAT is a glycoprotein primarily synthesized in liver cells, though it can also be produced by cells in the lining of the respiratory airways.
Its expression is controlled by two alleles present on chromosome 14 (an autosome).
The primary function of the protein is to protect the lungs from a protease (neutrophil elastase) secreted by white blood cells for immune response
\footnote{from \url{https://www.kamada.com/therapeutic_areas/alpha-1-antitrypsin-deficiency/}}.
It does so by binding to the protease.
Hence if there is a deficiency in functioning AAT, the lungs will be attacked by NE causing respiratory disorders such as emphysema.

\subsection{The deficiency AATD}
AATD is a gene disorder that is caused by abnormalities in the genes responsible for the production of AAT.
Such are caused by mutations.
\citeA{genengg} lists insertion and deletion mutations on the gene being contributors; however, the most common mutation is the missense from a point mutation causing a supposed glutamic acid to be replaced by a leucine amino acid.
Upon examination of the amino acids, glutamic acid (Glu) has a carboxylic acid in its R group making it acidic, while leucine (Leu) has an aliphatic R group.
This allele variant has been called the PiZ variant.
The acidic nature of Glu makes it charged under normal blood plasma pH, while Leu being aliphatic contributes no charge.
By changing the R group, a change in protein charge occurs, which in this case causes a reduction in the binding capability of the mutated protein to neutrophil elastase.
Another effect of the substitution is the ``destabilization'' of the resulting protein causing it to undergo ``polymerization'' which causes it not be secreted outsize the cell.
Resulting to an overall deficiency of AAT.

The genetic nature of the disease was also studied in participating families.
It has also been established that environmental conditions can affect the expressions of the PiZ variant.
The term used to describe this observation was ``gene-environment interaction'' resulting from epigenetics.
It was found that individuals homozygous for the allele did not need environmental stresses to contract respiratory disorders, while those heterozygous (containing normal variants) needed environmental stresses.
Another interesting result was also found: even if individuals have the same genotype and were subjected to the same environment, results were different.
This suggested that there are mechanisms other than genetics, the environment, and their interaction (epigenetics).
This was not highlighted in the study.

\section{Therapies}
\subsection{Augmentation therapy}
Since the deficiency in the amount of the protein causes diseases, a reasonable treatment is replenishment to normal amounts of the protein.
This is termed augmentation therapy.
It was deemed to be a feasible form of treatment given weekly infusions of doses of the protein.
Though it was found that testing for disease severity is impractical, the biochemical treatment (through doses) is sufficient such that it normalized the amount of AAT in lung plasma.

\subsection{Gene and stem cell therapy}
%% classical form of gene therapy
The classical form of gene therapy is the insertion of the normal gene into the affected cells, \textit{i.e.} those with the PiZ variant.
The studies reviewed used viral vectors to transfer genetic material.
It was found that adeno-associated viral vectors with functional ATT gene are capable of achiving higher levels of ATT in the lung plasma.
It was also noted that: it was less likely to induce inflammation compared to other viral vectors used.

There were proposed routes of administration (RoA) for the viruses: through the hepatic portal vein, through airways, and intramuscular injection.
Injection of the virus vectors through the hepatic portal vein seemed reasonable since the liver is where bulk of AAT is synthesized.
However, this was thought to be impractical if a patient needed many doses of injection to be treated.
The portal vein is difficult to inject material to being found within the liver
\footnote{Found through consultation of 3D anatomic model in \url{https://www.healthline.com/human-body-maps/portal-vein#1}}.
Use of nebulization for administering vectors through airways was recognized as a viable alternative.
The most successful route of administration among those reviewed was the intramuscular injection of vectors.
The direct addition of vectors to hepatic cells has been shown to be effective \textit{in vitro}, and \textit{in vivo} methods have yet to be developed and assessed.

The capability of treated stem cells to deliver corrected genes to many tissues has been studied on mice.
It was shown that it was effective, and clinical trials on human patients are to be expected.
One possible method is the injection of vectors into the bone marrow, where stem cells are produced.
These bring hope for advancement of treatment methods.
