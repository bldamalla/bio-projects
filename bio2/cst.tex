\chapter{Complex systems theory on adaptation and evolution (II)}

When studing complex traits (or traits, in general), it has become a reasonable assumption that there must be complex task or operation that requires such traits.
Because otherwise, such complex traits, requiring energy to maintain, would be pointless to have.
Evolutionary biologists deal with the problem of ``teleology'' in studying trait evolution \cite{Veloso2019}.
Some questions about the design of organisms and their complex traits can be raised: is there such a thing as a supernatural designer?
A similar question relating to teleology, but in a more applied aspect: is it possible to design nature the way humans, as social and intelligent beings, want?

The questions above can be answered in many different ways.
Since they will be answered in Part III of the collection, an exposition of concepts leading to the responses are presented in Part I.
Syntheses from observations in different biological and ecological systems supporting the introduction from Part I are presented in Part II.
Aside from the teleological questions, the axiom that biological and ecological systems are complex adaptive systems \cite{Dong2019} implies occurrence, given conditions, of catastrophe.
A small discussion on catastrophes will also be included in Part III.

\section{Complex systems in biology}
% in here just describe the interactions between agents

In this chapter, connections between complex systems theory and certain biological systems are introduced.
Specifically, some aspects of complexity in biological and ecological interactions are discussed.
The conclusions from which are crucial to the development of background for later chapters.

A common trait of complex systems is the non-linearity. % needs citation use Strogatz?
In simpler contexts, nonlinear interaction between different ``agents'' of a complex system depend is synonymous to feedback interaction.
Such complexity can arise in different levels of organization: from chemical feedback mechanisms within an organism or cell \cite{Chaves2019}, to multicellular interaction \cite{Veloso2017}, to community interactions in different trophic levels \cite{Seibold2018}.

\subsection{Biological interactions}
\subsubsection{Biochemical pathways of metabolites}
% look up those systems that are necessary for metabolites

There is a substantial amount of literature describing the different metabolic pathways and their regulation.
Among those is a study on the mathematical modeling of feedback and feedforward interactions controlled by genes \cite{Chaves2019}.

Biochemical metabolic pathways are a series of chemical reactions catalyzed by enzymes wherein products are used for some operations inside or outside the system.
An example of a metabolic pathway is the citric acid (Krebs cycle) cycle.
In the Krebs cycle, intermediates in the production of ATP are synthesized.
Syntheses pathways in the production of 2-Ketoglutarate, one of the important intermediates of the Krebs cycle, from xylose are studied using a newly developed algorithm \cite{Gupta2018}.
The study showed that there are different ways in producing the said intermediate.
This would mean that if such different pathways were to interfere with the citric acid cycle, feedback inhibition may occur \cite{Chaves2019}.
Conversely, the citric acid cycle may interefere with the ``different pathways''.

In the nonlinear dynamics model of \citeA{Chaves2019}, certain assumptions between interactions of intermediates of metabolite synthesis and enzymes used for metabolism were made.
In the study, metabolites in the specific metabolic system can interfere with synthesis in different parts of the pathway.
The following were taken into consideration:

\begin{enumerate}
    \item Metabolites regulating enzyme activity;
    \item Enzyme kinetics;
    \item Enzyme synthesis by the cell;
    \item Metabolites regulating enzyme synthesis;
    \item Concentration dilution as an effect of cell growth
\end{enumerate}

The reaction dynamics will not be discussed in the reports; however, the complexity and interdependence of the regulation agents have been shown.
Different graphs showing respective concentrations in certain time scales (not shown in the article) can be made using different parameters in respective domains.
It can be easily hypothesized that different cell parameters can lead to very varied results, as a result from complex systems theory. % still needs citation (probably Devaney)

\subsubsection{Developmental biology and epigenetic landscape}
% there is quite an extensive lit on the epi landscape apparently
% there are two in the "Complexity" folder and one in Firefox
% these two complex systems should be enough

\subsection{Ecological interactions}
% look up complex systems in higher scales
% do not yet focus on the problems, but rather on how they interact
% most of the work has been summarized in Dong2019 and Seibold 2018
\subsubsection{Population ecology}

% the onw with spatial heterogeneity and competition Dong2019

\subsubsection{Community ecology}

One of the most common nonlinear system models in community ecology is the Lotka-Volterra model, also known as the predator-prey model \cite{Seibold2018}.
Without mathematical modeling, the scenario can be set up as follows:

\begin{itemize}
    \item Consider two trophic levels: prey and predator
    \item Suppose there is relatively fewer predators to prey
    \item Suppose once the prey supply decreases, its consumption also decreases
    leading to eventual population growth through reproduction
    \item While prey supply decreases, predator population growth also decreases
    \item Finally, as prey population increases predator population follows suit
\end{itemize}

It is evident from the third item that the \emph{sustainability} of the complex system relies on the reproductive rate of prey species and consumption rate of predator species.
Suppose the prey species produces at a much slower rate compared to the consumption rate of the predator species.
A simulation of the above interaction will intuitively lead to the extinction of the prey species, thus requiring the predator species to occupy a new niche.
In other words, changes in environmental or species parameters have led to system death.

Anthropogenic interactions (human interventions) and their consequences, while considered as community ecological interactions, will be discussed in the last chapter of the reports.

\section{Self-organization and catastrophes in real systems}
% in here describe how interactions will lead to self-organization
% towards asymptotic ideals (?)

\subsection{Self-organization in physical systems}
\subsection{Self-organization in ecological systems}
\subsection{Bifurcations and catastrophes}

Bifurcations are a special term for significant changes in the way agents may asymptotically approach an optimal ``self-organized'' system.
These happen from significant changes in parameters dependent that control the properties of the agents.
In the figure below, a simple kind of bifucation is shown.
The saddle-node bifurcation is one of the simplest, yet most important bifurcation models.
In the bifurcation model, if the control parameter exceeds a threshold the loss of self-organization capabilities occurs and leads to system death (catastrophe).
%%% study this later
%% cite the appropriate author

In the case of intracellular systems, one such parameter may be related to the size of the cell and its use of resources.
This is an indirect conclusion from the above introduced study on metabolite-enzyme regulation study \cite{Chaves2019}.
Suppose there were mutations that allow for fast cell growth, requiring fast utilization of metabolites; however, other parameters such as the interaction of enzymes and substrates remain the same (\textit{ceteris paribus}).
Then there will come a point that the cell cannot sustain itself, and thus will result to its own destruction.
The study of how such parameters affect the system dynamics is in the realm of bifurcation theory, and will not be explored here (partly due to the lack of deeper knowledge of the researcher).

\section{Influences of self-organization on adaptation and evolution}
From the exposition of biological and ecological interactions between different agents in different
