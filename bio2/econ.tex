\chapter{Ecological economics and global climate change (II)}

\lipsum[34]

%% put some intro on the biodiversity crisis in here 
%% relate slightly to human niche construction: history, problem, and proposed solns

%% also introduce some economics: goods exchange and economic growth
%% also the movement towards ecological economics

\section{Human niche construction}
\subsection{Historical perspectives}
Throughout history, there have been important events that have altered the economic lifestyles of our species.
These include the Neolithic revolution and the Industrial revolution.
In the Neolithic revolution, man has discovered how to utilize agriculture to maintain a sedentary lifestyle.
Plants, livestock, and pollinators were used to increase the yields of agriculture.
There were many ``experiments'' on the artificial selection of animals, including the domestication for use as aids in farming and human consumption.
Such use and alteration of niches of species other than one's own to suit their needs is termed \emph{niche construction}
\footnote{from the Wikipedia page on niche construction}.
However, as remarked economies were stuck in what was called a ``Malthusian trap'' (\textit{i.e.} no economic growth)
\footnote{from the Sixteen Page Economic History of the World, Princeton University Press}.

The Industrial revolution marked the start of rapid economic growth.
This meant that more resources will have to be retrieved and processed for human consumption.
This further implies the creation of goods and services from natural resources.
Greater volumes of material processing implies that more waste is produced, and such can lead to large amounts of environmental polluters.
One of the effects of which is \emph{global climate change}.

\subsection{Pollution and global climate change}
One of the problems faced by society nowadays is global climate change, with anthropogenic greenhouse gases as primary suspects.

%% put here summaries from Chapter 4-1

\subsection{Proposed solutions}
%% low carbon
%% resource recycling
%% "nature-friendly"
%% reliable tools
%%% also include a short summary of chapter 7

\section{Ecological economics}
\subsection{Importance and economics of abiotic resources}
%% summarize the subsection on abiotic resources
%%% write a paragraph each for the economics and importance of abiotic resources

\subsection{Importance and economics of biotic resources}
%% summarize the subsection on biotic resources
%%% write a paragraph each for the economics and importancce of biotic resources

\paragraph{Synthesis and conclusion}
