\chapter{Global climate change (II)}

%% put some intro on the biodiversity crisis in here 
%% relate slightly to human niche construction: history, problem, and proposed solns

%% also introduce some economics: goods exchange and economic growth
%% also the movement towards ecological economics

\section{Human niche construction}
\subsection{Historical perspectives}
Throughout history, there have been important events that have altered the economic lifestyles of our species.
These include the Neolithic revolution and the Industrial revolution.
In the Neolithic revolution, man has discovered how to utilize agriculture to maintain a sedentary lifestyle.
Plants, livestock, and pollinators were used to increase the yields of agriculture.
There were many ``experiments'' on the artificial selection of animals, including the domestication for use as aids in farming and human consumption.
Such use and alteration of niches of species other than one's own to suit their needs is termed \emph{niche construction}
\footnote{from \url{https://en.wikipedia.org/wiki/Niche_construction#Humans}}.
However, as remarked economies were stuck in what was called a ``Malthusian trap'' (\textit{i.e.} no economic growth)
\footnote{from \url{http://assets.press.princeton.edu/chapters/s8461.pdf}}.

The Industrial revolution marked the start of rapid economic growth.
This meant that more resources will have to be retrieved and processed for human consumption.
This further implies the creation of goods and services from natural resources.
Greater volumes of material processing implies that more waste is produced, and such can lead to large amounts of environmental polluters.
One of the effects of which is \emph{global climate change}.
As noted in the previous section on mass extinctions there are two possible outcomes, either there will be a rapid mass extinction similar to the Cretaceous-Paleogene boundary or a period of diversification as a result of pseudoextinction similar to the Paleocene-Eoecene boundary.

\subsection{Pollution and global climate change}
One of the problems faced by society nowadays is global climate change, with anthropogenic greenhouse gases as primary suspects.
It has been said that human activities constitute to about $90\%$ the global greenhouse gas flux \cite{introclim}.
Because of this various considerations on the next actions will have to be made.

Anthropogenic global climate change (GCC) is a problem with many dimensions affecting different aspects of ecosystems and societies.
Two of the most important factors interacting in GCC are the global system (biosphere) from which humans as an animals species obtain our necessities and human systems including philosphies, psychologies, and politics \cite{introclim}.
These subsystems are dependent on each other, so it is necessary to understand and act.
GCC, having to manifest its effects in the future 
\footnote{possibly to a much greater intensity as is experienced now}
produces inequity between generations.
To help in understanding the extent of climate change in later generations simulations are being done; however, they are not necessarily reliable (has a non-deterministic nature) since weather is chaotic.
To further complicate issues, changes in economic situations especially in producer and consumer dynamics can also affect climate \cite{introclim}.

%% put here summaries from Chapter 4-1

Water circulation (hydrologic cycle) is an important part of the geochemical cycles \cite{overclim}.
The effects of global climate change on the hydrologic cycle have been studied.
It has been found that GCC has effects on average temperature, amount of precipitation, evaporation, and the sea level.
Runoff was observed to be and predicted to increase in some areas.
Glaciers are threatened.
The frequency and intensity of drought and precipitation were found to increase in simulations.
Finally, water quality is also affected by causing ocean acidification and salination.
This further decreases the amount of freshwater available for consumption.

Organisms in ecosystems are also affected.
There are observations of biodiversity losses in tundra, boreal forest, and coral reef ecosystems \cite{overclim}.
Many more environments are stressed.
It is estimated that there will be larger biodiversity losses since $20-30\%$ of studied species are at high risk of extinction.

\subsection{Proposed solutions}
%% low carbon
%% resource recycling
%% "nature-friendly"
%% reliable tools
%%% also include a short summary of chapter 7
Generally, two forms of solutions are proposed and implemented: solutions to mitigate the effects of GCC and to adapt to GCC.
Mitigation measures opt for the reduction of emissions and increase in absorption of greenhouse gases.
These are done by the establishment of mitigation policies, but the effects of these policies are only expected to be manifested for the next several decades \cite{overclim}.
Adaptation measures on the other hand increase the ``preparedness'' of humans and ecosystems for the adverse effects of GCC.
Since the effects of mitigation effects are not directly manifested, adaptation measures are also important to improve on.

One thing to note is that the effects of climate change cannot be suppressed and that impacts are certain to manifest, and this problem is more serious in developing countries \cite{overclim, introclim}.
As noted in Chapter 1, it is important to study the factors affecting the complexity of the problems in order to make better decisions, in the case of ecological complexity to help in adapting to GCC.
With this in mind, new paradigms in economics are being developed to create new schools of thought under the name ``ecological economics''.

\section{Ecological economics}
\paragraph{On the law of scarcity}
One of the most fundamental laws in economics is the law on scarcity which states that there will never be enough resources to satisfy the needs of every person.
Neoclassical economics (NCE, the standard economic paradigm) says that there is no limit to the growth of economies (through production).
This contradicts with the law of scarcity, upon which most of microeconomic theory is built upon.
Ecological economics (EE) on the other hand, proposes an \emph{optimal scale} where economic growth can be limited \cite[p.15-18]{daly}.
The limitation of economic growth, while focusing on development is crucial for maintaining the sustainability of economies and ecologies.

\paragraph{The optimal scale}
Utility is can be defined as the amount of satisfaction one receives upon consumption of the good.
This is usually considered in modeling consumer/producer dynamics in microeconomic NCE, but it is not considered in macroeconomics.
\citeA[p.15-18]{daly} propose a concept of disutility which is a form of negative satisfaction caused by a reduction of ``ecological welfare''.
The optimal scale exists when satisfaction is the same as dissatisfaction.

\paragraph{Synthesis}
Due to the rapic increase in greenhouse gas flux primarily caused by human niche construction, global climate change is observed.
Similar to the most recent mass extinction and the Paleocene-Eoecene thermal maximum (PETM), the average global temperature increases rapidly \cite{Keller2018}.
This means that climate change may proceed to two extremes: mass extinction and pseudoextinction.
In both cases, there are massive biodiversity losses.
Global climate change has already done numerous effects on abiotic (water) and biotic (ecosystem) resources.
This has serious impacts on the economics within ecosystems and of man-made economies as well.
Three kinds of solutions are being studied: adaptation, mitigation \cite{overclim}, and philosophical strategies \cite{daly}.
Though GCC is a natural phenomenon, knowing the intricate relationships between ecologies, economies, and societies, the manifestation of effects may be lessened.
