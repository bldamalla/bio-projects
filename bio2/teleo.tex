\chapter{Teleological arguments and intelligent design (II)}

As has been exposed in Part I, selection can be classified into two groups: natural and artificial selection.
The two concepts will be reintroduced in this chapter to give a possibly sound discussion on the validity of natural selection.
The infallability and logical flaws \footnotemark in the main arguments of intelligent design are also discussed.
\footnotetext{This is based on personal opinion and knowledge.}
Other arguments (and respective counterarguments) against natural selection are also exposed.

%% reintroduce darwin again
\section{Selection}
\subsection{Natural selection}
Natural selection is a concept wherein a species variant that is considered to be more fit for the environment is selected for its characteristics.
Charles Darwin has thought that natural selection is one of the major mechanisms for speciation and the evolution of traits.
It has been shown that it can operate on the different scales of ecological organization, as in the case of HIV evolution \cite{Ridley} and industrial melanism \cite{biomain, Ridley}.

Though the mechanisms for some natural selection occurrences are not easily determined as in the case for industrial melanism, correlations between variables have been established and has shown that natural selection persists \cite{Ridley}.
An example of such correlation is between the frequency of each variant and the local environment.
It has been shown that the melanic variant is considered to be more fit in the polluted environment, while the normal variant is more fit in the less polluted variant.
Given this information, there can be a lot of factors that can cause the selection: first is predation (camouflage as a defense mechanism), a second is toxin immunity (effect of pleiotropy).
\footnote{from \url{https://en.wikipedia.org/wiki/Industrial_melanism}}
Experiments can be made to determine the mechanism for selection.

\subsection{Artificial selection}
Artificial selection was done by humans as a part of niche occupation in order to adapt to the environment.
Examples of experiments are on increasing the milk yield of cows \cite{Ridley} and on the taming of a fox \cite{biomain}.
Increasing cow milk yield has purposes in economics and agriculture, while fox domestication is for aid.
Both experiments have shown that trait selection increases the frequency of the selected trait \cite{Ridley}.

Artificial selection would not be possible if natural selection does not happen.
The simplest argument would be that they follow the exact same mechanism but at different rates.
Since artificial selection can be controlled, the most of the parameters in which it operates can be manipulated to give some expected results.
Though there may be some exceptions as in the change in appearance of the domesticated fox described above (a [hypothesized] result of pleiotropy) \cite{biomain}.

\subsection{Misunderstanding maths}
One of the common misconceptions in inferential statistics is the belief that correlations may be causations.

\paragraph{Correlations and causations}
From above, one interesting case is of pleiotropy in peppered moths.
Suppose there is a gene that has effects on wing color and pollutant resistance.
Experiments were conducted such that wing color is related to the mechanism of selection.
As a hypothesis on selection mechanism, pleiotropy can be justified or falsified through further studies on the peppered moth genome.

Even without concrete examples, the statement can still be proven false.
Strong correlation between two variables can be observed if changes in one of the variables produces a determinate change in another.
Correlation is a statistical description, while causation is a logical description.
This means that correlations \emph{always} have two-way relationships (and \textit{vice-versa}), it does not matter which is a dependent or independent variable
\footnote{though hypotheses should properly state which makes the more logical sense}, while
a causation may not work both ways (the converse of the statement is false).
Take for example the following statements: ``very high sugar intake causes diabetes'' and ``very high sugar intake has strong relationships with signs found in diabetes''.
The difference in nuance is precisely the difference between causation and correlation.

%% design and intelligent design

\section{Intelligent design}
\subsection{Irreducible complexity}
Irreducible complexity is an argument of intelligent design where adaptations are fine tuned to operate with each other in the sense that if one component were missing, the system would collapse.
This can be observed in some developing blastula, wherein if one cell has been taken away, then animal development will stop.
\footnote{from one of the lectures in Bio II on animal diversity}
The argument has been related to the \emph{Watchmaker analogy}.

\paragraph{The Watchmaker Analogy}
In the same way that a functioning watch has been designed by its intelligent craftsman wherein all of its components are doing their respective functions, the Universe must also have been created by an intelligent being.
\footnote{from \url{https://en.wikipedia.org/wiki/Intelligent_design}; some words were changed and omitted}

There are two keywords to note from the analogy: design and function.
Structure and function cannot be entirely isolated from each other \cite{anaphy}, and hence it has then become reasonable to use the terms together.
In studying the logic of design, each part present in the system (\textit{e.g.} in organisms), must have uses if it is included.
Obviously, the observation of structures \emph{not having function} immediately breaks the watchmaker analogy.
Vestigial organs/structures are such structures.
Examples of which are pelvic bones in whales and hindlimbs in snakes \cite{Ridley}, which can be explained through natural selection and genetic drift.

Another implication of the argument of irreducible complexity is the observation of emergent properties.
If the rules of mathematics are considered as universally true, then so would be the observation of emergent properties in some complex adaptive systems.
An example of application of maths is the study of \cite{Veloso2017} on self-regulation in multicellular systems, wherein individual cells adapt in the presence of other cells.

\subsection{Specified complexity}
Specified complexity is observed when information of apparently very low probability has been observed.
\footnote{from \url{https://en.wikipedia.org/wiki/Intelligent_design}}
In the sense that the probabilities make the existence of such information improbable (but not impossible).
An example of which is the genetic code in the genes that code for proteins.
Suppose there is an intelligent deity that has made the genetic code so that it codes complex systems of proteins with emergent properties.
Was the intelligent being also responsible for mistakes in mutations causing genetic drift for natural selection to occur?

\paragraph{Synthesis}
In both arguments, irreducible and specified complexity had logical weak spots, and that the second argument is not falsifiable.
Natural selection is a statement of correlation on the nature of interaction between agents, while intelligent design is a statement of causation of design and function.
As said, correlations and causations have different nuances and hence are not equivalent.
In any case, conclusions from correlations are bound by certain mathematical laws, and that such laws agree with Darwin's natural selection.
On the other hand, intelligent design seeked a logical causation that had unjustifiable holes.
