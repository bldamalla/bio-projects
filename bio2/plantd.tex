\chapter{Plant defense mechanisms (II)}

Defense mechanisms are traits used by certain species to prevent others from attacking them.
These can be thought of as adaptations for survival.
Being adaptations, such are results of complexities in the growth and development of the individual and their interactions with the environment.
Relating to relative fitness, plants with more successful defense mechanisms, consequently leaving more progeny, are considered to be more fit. % dont forget to add footnote on the sentence
Purposefully leading to the discussion on teleology as a point of view in studying evolution, defense mechanisms provide a wide array of expression and questions.

\section{General overview}
As given by \citeA[p.792-802]{biomain}, there are different kinds of plant defense mechanisms for different kinds of situations.
These include: physical defenses such as protection of the inner tissues with the epidermal cells,
chemical defenses such as producing toxic substances to herbivores (including humans),
pathogen recognition and plant cell death to prevent spread, and
signaling pathways for plant immunity.
As said, plant defense mechanisms can be very complex.

\paragraph{Physical defenses}
The epidermal layer of the plant serves as a layer of protection of the ``internal organs'' of the plant.
Plants develop strong bark and waxy cuticles for structural integrity and maintenance of internal environments (by preventing water loss).
Those structures also double as protection.
The strong bark of the plant makes the material difficult to mechanically digest, requiring a lot of energy, and thus may not be an efficient source of nutrition for some organisms.
The waxy cuticles, being thick layers of lipids, make it difficult for pathogens to infect the internal structures of the plant.
Some plants also develop thorns from lateral meristems to prevent large herbivores from eating the plant.

\paragraph{Chemical defenses through secondary metabolites}
Many plants produce chemicals that can potentially kill herbivores such as poisonous substances.
These include cyanogens which produce cyanide when metabolized by the herbivore.
Plant secondary metabolites can also change the behavior of the herbivores.
Such include caffeine, cocaine, and morphine.
Humans have learned to use the metabolites in medicine by observing herbivores under effects of the drugs \cite[p.794-796]{biomain}.

\paragraph{Pathogen recognition and defenses}
When plants get wounded, they become susceptible to pathogens such as bacteria invading the interior of the plant.
Wounded leaves produce a peptide sequence called systemin which travels through the vascular system of the plant.
This leads to the production of jasmonic acid activating the transcription of defense genes.
It was also observed that plants have a kind of ``immunity'' by recognizing pathogens, but the mechanisms are quite different from animal immune systems \cite[p.799-800]{biomain}.

\section{Evolutionary effects on herbivory and the ecosystem}
As has been said, being adaptations, plant defense mechanisms are crucial to protect the plant from ecological stresses such as herbivory.
The interactions of herbivores with the plant with two types of defenses, as studied by \citeA{Loeuille2018}, have been found to have quantitative effects on the fitness of the plant with respect to the herbivore environment.
The theoretical simulations, variables and parameters will be reviewed in this section.

%% describe here what are the two types of defenses being used in the study
\subsection{Summary of the study}
With the general overview of general plant defense mechanisms in the previous section, such can be divided into two categories: \emph{quantitative} and \emph{qualitative} defenses.
Quantitative defenses are those that require energy and dose to have effects on the herbivore or pathogen that threatens the plant.
Qualitative defense on the other hand only determines herbivore deterrence or attraction by the presence or absence of the characteristics.
As was assumed in the aforecited study, these defense mechanism \emph{bundles} have no energy cost for the plant.

%% put here the objectives of the study
The objectives of the study were to determine the effects of plant defense strategies to herbivory,
to simulate whether the evolution of the defense strategies has a role in the diversification of strategies,
and to study the effects of defense evolution in ecological diversity and dynamics.

The variables used in the development of the simulations were the following:

\begin{enumerate}
    \item Plant and herbivore respective biomasses
    \item Plant quantitative and qualitative defenses
    \item The ``preference'' and ``generalism'' of the herbivore to the plant defenses
\end{enumerate}

On the other hand, the parameters used were:

\begin{enumerate}
    \item Plant basal carrying capacity
    \item Conversion efficiency
    \item Plant growth rate
    \item ``Benefits'' of plant quantitative defenses in terms of reduced plant consumption
    \item ``Costs'' of quantitative defenses
    \item Herbivore mortality rate
    \item Basal herbivore consumption
    \item Variance of competition kernel
\end{enumerate}

The Lotka-Volterra system \footnote{see Chapter 1 for an overview} was used in modeling population dynamics.
The choice of the model for the description of population dynamics is reasonable since, plants act as prey and herbivores act as predators.
The biomass growth of both system actors (plant and herbivore) were based on the growth of each, their mortalities, competition, and assumed basal carrying capacity.
Since the variables are dynamic (they change with respect to each other), the proposed representation seemed adequate for the system.

The evolution of the plant defense strategies were said to be from the pressure exerted by herbivory, which relates to the fitness of the plants (resident and mutant),
and was modeled based on a ``canonical equation of adaptive dynamics''.
Fitness was defined as the rate at which the biomass grows given the defense traits of the plants present.
This accounts for the ``evolutionary'' definition of fitness \cite{Ridley}.
The canonical equation regards (small) mutation rates and the fitness benefit of slight changes in characteristics.

At the end of the simulations, it was found that the evolution of quantitative defense traits helps to maintain or increase diversity of defense traits, 
while the evolution of qualitative traits have detrimental effects on the coexistence of plants and herbivores.

Though the study was successful in achieving their objectives, there were parts of the model that were unclear \footnotemark.

\footnotetext{This is an opinion on the parts of the article that were unclear to me.
Unfortunately, I have not found answers anywhere else, so I have decided to put them in the next section.}

\subsection{Critiques and comments on the study}

The critiques on the study are mainly based from the assumptions of the researchers \footnotemark.
\footnotetext{This subsection is primarily based on my knowledge on the topic.}
The first concern is related to the use of a single number to represent the defense mechanisms of the plant.
As noted above, plants have developed a wide variety of defense mechanisms against herbivory including physical and chemical defenses.
Quite similarly to human behavior, herbivores in the study were assumed to be able to ascertain preferences in the kinds of plants they ingest.
The herbivore preferences were also represented as numbers.
If the number representing the defense trait of a plant variant B is farther than that of another plant A, then the herbivore would prefer eating plant A.

Consider the following thought experiment
\footnote{I have devised this on my own based on consumer economics}.
Given two defense traits that the herbivore is averse to, it will not want to be exposed more to them but rather receive less of them.
Suppose there exists a set of combination of traits such that the herbivore does not prefer any one of those.
In other words, the herbivore is indifferent to those combinations.

Then, even if the defense traits of the plants are different (thus represented by different numbers), the ``preference'' of the herbivore should not change (should still be represented by the same number).
With this scenario, then there is a contradiction with the methodology of the researcher, \textit{i.e.} it would be impossible to study the evolution of plant and herbivore traits.

The problem is observed to arise when the multidimensional problem of trait preference is simplified into a single non-operational representation.
Of course there are a lot of dimension reduction studies in statistics\footnote{such as principal component analysis and partial least squares regression} and in other fields.
Yet such studies admit to the loss of ``explainability'' at the cost of modeling capability.

The second concern, related to the first, is on the generality of the assumptions used in modeling.
The study does not go anywhere specific and that may be the reason why the equations, parameters, and variables appeared too vague.
It is understandable that it is for the sake of simplicity and getting the desired results as were described in the objectives.
But it hides the logical structure that underlies the mechanism of defenses.
